%% Generated by Sphinx.
\def\sphinxdocclass{report}
\documentclass[letterpaper,10pt,english]{sphinxmanual}
\ifdefined\pdfpxdimen
   \let\sphinxpxdimen\pdfpxdimen\else\newdimen\sphinxpxdimen
\fi \sphinxpxdimen=.75bp\relax

\PassOptionsToPackage{warn}{textcomp}
\usepackage[utf8]{inputenc}
\ifdefined\DeclareUnicodeCharacter
% support both utf8 and utf8x syntaxes
\edef\sphinxdqmaybe{\ifdefined\DeclareUnicodeCharacterAsOptional\string"\fi}
  \DeclareUnicodeCharacter{\sphinxdqmaybe00A0}{\nobreakspace}
  \DeclareUnicodeCharacter{\sphinxdqmaybe2500}{\sphinxunichar{2500}}
  \DeclareUnicodeCharacter{\sphinxdqmaybe2502}{\sphinxunichar{2502}}
  \DeclareUnicodeCharacter{\sphinxdqmaybe2514}{\sphinxunichar{2514}}
  \DeclareUnicodeCharacter{\sphinxdqmaybe251C}{\sphinxunichar{251C}}
  \DeclareUnicodeCharacter{\sphinxdqmaybe2572}{\textbackslash}
\fi
\usepackage{cmap}
\usepackage[T1]{fontenc}
\usepackage{amsmath,amssymb,amstext}
\usepackage{babel}
\usepackage{times}
\usepackage[Bjarne]{fncychap}
\usepackage{sphinx}

\fvset{fontsize=\small}
\usepackage{geometry}

% Include hyperref last.
\usepackage{hyperref}
% Fix anchor placement for figures with captions.
\usepackage{hypcap}% it must be loaded after hyperref.
% Set up styles of URL: it should be placed after hyperref.
\urlstyle{same}
\addto\captionsenglish{\renewcommand{\contentsname}{Contents:}}

\addto\captionsenglish{\renewcommand{\figurename}{Fig.}}
\addto\captionsenglish{\renewcommand{\tablename}{Table}}
\addto\captionsenglish{\renewcommand{\literalblockname}{Listing}}

\addto\captionsenglish{\renewcommand{\literalblockcontinuedname}{continued from previous page}}
\addto\captionsenglish{\renewcommand{\literalblockcontinuesname}{continues on next page}}
\addto\captionsenglish{\renewcommand{\sphinxnonalphabeticalgroupname}{Non-alphabetical}}
\addto\captionsenglish{\renewcommand{\sphinxsymbolsname}{Symbols}}
\addto\captionsenglish{\renewcommand{\sphinxnumbersname}{Numbers}}

\addto\extrasenglish{\def\pageautorefname{page}}

\setcounter{tocdepth}{1}



\title{Laboratory Documentation}
\date{Feb 07, 2019}
\release{0.1}
\author{Sam Jennings}
\newcommand{\sphinxlogo}{\vbox{}}
\renewcommand{\releasename}{Release}
\makeindex
\begin{document}

\pagestyle{empty}
\maketitle
\pagestyle{plain}
\sphinxtableofcontents
\pagestyle{normal}
\phantomsection\label{\detokenize{index::doc}}



\chapter{Module Laboratory}
\label{\detokenize{laboratory:module-laboratory}}\label{\detokenize{laboratory:module-laboratory}}\label{\detokenize{laboratory::doc}}\index{laboratory (module)@\spxentry{laboratory}\spxextra{module}}\index{Setup (class in laboratory)@\spxentry{Setup}\spxextra{class in laboratory}}

\begin{fulllineitems}
\phantomsection\label{\detokenize{laboratory:laboratory.Setup}}\pysiglinewithargsret{\sphinxbfcode{\sphinxupquote{class }}\sphinxcode{\sphinxupquote{laboratory.}}\sphinxbfcode{\sphinxupquote{Setup}}}{\emph{filename=None}, \emph{debug=False}}{}
Sets up the laboratory


\begin{savenotes}\sphinxattablestart
\centering
\begin{tabulary}{\linewidth}[t]{|T|T|}
\hline
\sphinxstyletheadfamily 
Attributes
&\sphinxstyletheadfamily 
Description
\\
\hline
data
&
houses lab data during measurements or after parseing
\\
\hline
logger
&
creates a logger for error reporting
\\
\hline
plot
&
contains different plotting tools for data visualisation
\\
\hline
command
&
contains the drivers for controlling instrumentation
\\
\hline
\end{tabulary}
\par
\sphinxattableend\end{savenotes}


\begin{savenotes}\sphinxattablestart
\centering
\begin{tabulary}{\linewidth}[t]{|T|T|}
\hline
\sphinxstyletheadfamily 
Public Methods
&\sphinxstyletheadfamily 
Description
\\
\hline
device\_status
&
checks the status of all connected devices
\\
\hline
get\_gas
&
retrieves and saves data from a single mass flow controller
\\
\hline
get\_impedance
&
retrieves and saves complex impedance data from the LCR meter
\\
\hline
get\_temp
&
retrieves and saves temperature data from the furnace
\\
\hline
get\_thermo
&
retrieves and saves thermopower data from the DAQ
\\
\hline
load\_data
&
will parse a datafile and store in ‘data’ structure for post-
processing and visualisation
\\
\hline
load\_frequencies
&
loads a set of frequencies into Data object
\\
\hline
load\_instruments
&
connects to all available instruments
\\
\hline
run
&
begins a new set of laboratory measurements
\\
\hline
\end{tabulary}
\par
\sphinxattableend\end{savenotes}


\begin{savenotes}\sphinxattablestart
\centering
\begin{tabulary}{\linewidth}[t]{|T|T|}
\hline
\sphinxstyletheadfamily 
Private Methods
&\sphinxstyletheadfamily 
Description
\\
\hline
\_break\_loop
&
determines when the main measurement loop should break and start a
new step
\\
\hline
\_count\_down
&
controls the count down timer displayed between measurements
\\
\hline
\_measurement\_loop
&
main measurement loop of the program that retrieves and saves
data
\\
\hline
\_progress\_bar
&
controls the progress bar displayed during measurements
\\
\hline
\end{tabulary}
\par
\sphinxattableend\end{savenotes}
\begin{quote}\begin{description}
\item[{Example}] \leavevmode
\end{description}\end{quote}

\fvset{hllines={, ,}}%
\begin{sphinxVerbatim}[commandchars=\\\{\}]
\PYG{g+gp}{\PYGZgt{}\PYGZgt{}\PYGZgt{} }\PYG{k+kn}{import} \PYG{n+nn}{Laboratory}
\PYG{g+gp}{\PYGZgt{}\PYGZgt{}\PYGZgt{} }\PYG{n}{lab} \PYG{o}{=} \PYG{n}{Laboratory}\PYG{o}{.}\PYG{n}{Setup}\PYG{p}{(}\PYG{p}{)}
\PYG{g+gp}{\PYGZgt{}\PYGZgt{}\PYGZgt{} }\PYG{n}{lab}\PYG{o}{.}\PYG{n}{run}\PYG{p}{(}\PYG{l+s+s1}{\PYGZsq{}}\PYG{l+s+s1}{some\PYGZus{}controlfile}\PYG{l+s+s1}{\PYGZsq{}}\PYG{p}{)}
\end{sphinxVerbatim}
\index{\_break\_loop() (laboratory.Setup method)@\spxentry{\_break\_loop()}\spxextra{laboratory.Setup method}}

\begin{fulllineitems}
\phantomsection\label{\detokenize{laboratory:laboratory.Setup._break_loop}}\pysiglinewithargsret{\sphinxbfcode{\sphinxupquote{\_break\_loop}}}{\emph{step}, \emph{loop\_start}}{}
Checks whether the main measurements loop should be broken in order to proceed to the next step. If temperature is increasing the loop will break once T-indicated exceeds the target temperature. If temperature is decreasing, the loop will break when T-indicated is within 5 degrees of the target. If temperature is holding, the loop will break when the hold time specified by step.hold\_length is exceeded.
\begin{quote}\begin{description}
\item[{Parameters}] \leavevmode\begin{itemize}
\item {} 
\sphinxstyleliteralstrong{\sphinxupquote{step}} \textendash{} current measurement step

\item {} 
\sphinxstyleliteralstrong{\sphinxupquote{Tind}} (\sphinxstyleliteralemphasis{\sphinxupquote{float}}) \textendash{} current indicated temperature on furnace

\item {} 
\sphinxstyleliteralstrong{\sphinxupquote{loop\_start}} (\sphinxstyleliteralemphasis{\sphinxupquote{float}}\sphinxstyleliteralemphasis{\sphinxupquote{ (}}\sphinxstyleliteralemphasis{\sphinxupquote{time.time}}\sphinxstyleliteralemphasis{\sphinxupquote{(}}\sphinxstyleliteralemphasis{\sphinxupquote{)}}\sphinxstyleliteralemphasis{\sphinxupquote{)}}) \textendash{} start time of the current measurement cycle

\end{itemize}

\end{description}\end{quote}

\end{fulllineitems}

\index{\_count\_down() (laboratory.Setup method)@\spxentry{\_count\_down()}\spxextra{laboratory.Setup method}}

\begin{fulllineitems}
\phantomsection\label{\detokenize{laboratory:laboratory.Setup._count_down}}\pysiglinewithargsret{\sphinxbfcode{\sphinxupquote{\_count\_down}}}{\emph{start}, \emph{interval}, \emph{time\_remaining=1}}{}
Controls the count down until next measurement cycle
\begin{quote}\begin{description}
\item[{Parameters}] \leavevmode
\sphinxstyleliteralstrong{\sphinxupquote{interval}} (\sphinxstyleliteralemphasis{\sphinxupquote{float/int}}) \textendash{} time in seconds remaining until next measurement

\end{description}\end{quote}

\end{fulllineitems}

\index{\_measurement\_loop() (laboratory.Setup method)@\spxentry{\_measurement\_loop()}\spxextra{laboratory.Setup method}}

\begin{fulllineitems}
\phantomsection\label{\detokenize{laboratory:laboratory.Setup._measurement_loop}}\pysiglinewithargsret{\sphinxbfcode{\sphinxupquote{\_measurement\_loop}}}{\emph{step}}{}
This is the main measurement loop of the program. All data is accessed and saved from within this loop
\begin{quote}\begin{description}
\item[{Parameters}] \leavevmode
\sphinxstyleliteralstrong{\sphinxupquote{step}} \textendash{} a single row from the control file

\end{description}\end{quote}

\end{fulllineitems}

\index{\_progress\_bar() (laboratory.Setup method)@\spxentry{\_progress\_bar()}\spxextra{laboratory.Setup method}}

\begin{fulllineitems}
\phantomsection\label{\detokenize{laboratory:laboratory.Setup._progress_bar}}\pysiglinewithargsret{\sphinxbfcode{\sphinxupquote{\_progress\_bar}}}{\emph{iteration}, \emph{message}, \emph{decimals=0}, \emph{bar\_length=25}}{}
Creates a terminal progress bar
\begin{quote}\begin{description}
\item[{Parameters}] \leavevmode\begin{itemize}
\item {} 
\sphinxstyleliteralstrong{\sphinxupquote{iteration}} \textendash{} iteration number

\item {} 
\sphinxstyleliteralstrong{\sphinxupquote{message}} (\sphinxstyleliteralemphasis{\sphinxupquote{str}}) \textendash{} message to be displayed on the right of the progress bar

\end{itemize}

\end{description}\end{quote}

\end{fulllineitems}

\index{delayed\_start (laboratory.Setup attribute)@\spxentry{delayed\_start}\spxextra{laboratory.Setup attribute}}

\begin{fulllineitems}
\phantomsection\label{\detokenize{laboratory:laboratory.Setup.delayed_start}}\pysigline{\sphinxbfcode{\sphinxupquote{delayed\_start}}}
Starts the experiment at a given time the next day. Can be set to any 24 hour time in string format.
\begin{quote}\begin{description}
\item[{Example}] \leavevmode
\end{description}\end{quote}

\fvset{hllines={, ,}}%
\begin{sphinxVerbatim}[commandchars=\\\{\}]
\PYG{g+gp}{\PYGZgt{}\PYGZgt{}\PYGZgt{} }\PYG{k+kn}{import} \PYG{n+nn}{Laboratory}
\PYG{g+gp}{\PYGZgt{}\PYGZgt{}\PYGZgt{} }\PYG{n}{lab} \PYG{o}{=} \PYG{n}{Laboratory}\PYG{o}{.}\PYG{n}{Setup}\PYG{p}{(}\PYG{p}{)}
\PYG{g+gp}{\PYGZgt{}\PYGZgt{}\PYGZgt{} }\PYG{n}{lab}\PYG{o}{.}\PYG{n}{delayed\PYGZus{}start} \PYG{o}{=} \PYG{l+s+s1}{\PYGZsq{}}\PYG{l+s+s1}{0900}\PYG{l+s+s1}{\PYGZsq{}} \PYG{c+c1}{\PYGZsh{}start at 9am the following day}
\PYG{g+gp}{\PYGZgt{}\PYGZgt{}\PYGZgt{} }\PYG{n}{lab}\PYG{o}{.}\PYG{n}{run}\PYG{p}{(}\PYG{l+s+s1}{\PYGZsq{}}\PYG{l+s+s1}{some\PYGZus{}controlfile}\PYG{l+s+s1}{\PYGZsq{}}\PYG{p}{)}
\end{sphinxVerbatim}

\end{fulllineitems}

\index{device\_status() (laboratory.Setup method)@\spxentry{device\_status()}\spxextra{laboratory.Setup method}}

\begin{fulllineitems}
\phantomsection\label{\detokenize{laboratory:laboratory.Setup.device_status}}\pysiglinewithargsret{\sphinxbfcode{\sphinxupquote{device\_status}}}{}{}
Checks the status of all devices. If desired, this function can send an email when something has become disconnected
\begin{quote}\begin{description}
\item[{Returns}] \leavevmode
True if all devices are connected and False if any are disconnected

\item[{Return type}] \leavevmode
Boolean

\end{description}\end{quote}

\end{fulllineitems}

\index{get\_gas() (laboratory.Setup method)@\spxentry{get\_gas()}\spxextra{laboratory.Setup method}}

\begin{fulllineitems}
\phantomsection\label{\detokenize{laboratory:laboratory.Setup.get_gas}}\pysiglinewithargsret{\sphinxbfcode{\sphinxupquote{get\_gas}}}{\emph{gas\_type}}{}
Gets data from the mass flow controller specified by gas\_type and saves to Data structure and file
\begin{quote}\begin{description}
\item[{Parameters}] \leavevmode
\sphinxstyleliteralstrong{\sphinxupquote{gas\_type}} (\sphinxstyleliteralemphasis{\sphinxupquote{str}}) \textendash{} type of gas to use when calculating ratio (either ‘h2’ or ‘co’)

\item[{Returns}] \leavevmode
{[}mass\_flow, pressure, temperature, volumetric\_flow, setpoint{]}

\item[{Return type}] \leavevmode
list

\end{description}\end{quote}

\end{fulllineitems}

\index{get\_impedance() (laboratory.Setup method)@\spxentry{get\_impedance()}\spxextra{laboratory.Setup method}}

\begin{fulllineitems}
\phantomsection\label{\detokenize{laboratory:laboratory.Setup.get_impedance}}\pysiglinewithargsret{\sphinxbfcode{\sphinxupquote{get\_impedance}}}{}{}
Sets up the lcr meter and retrieves complex impedance data at all frequencies specified by Data.freq. Data is saved in Data.imp.z and Data.imp.theta as a list of length Data.freq. Values are also saved to the data file.

\end{fulllineitems}

\index{get\_temp() (laboratory.Setup method)@\spxentry{get\_temp()}\spxextra{laboratory.Setup method}}

\begin{fulllineitems}
\phantomsection\label{\detokenize{laboratory:laboratory.Setup.get_temp}}\pysiglinewithargsret{\sphinxbfcode{\sphinxupquote{get\_temp}}}{\emph{target}}{}
Retrieves the indicated temperature of the furnace and saves to Data structure and file

\begin{sphinxadmonition}{note}{Note:}
this is the temperature indicated by the furnace, not the temperature of the sample
\end{sphinxadmonition}
\begin{quote}\begin{description}
\item[{Parameters}] \leavevmode
\sphinxstyleliteralstrong{\sphinxupquote{target}} (\sphinxstyleliteralemphasis{\sphinxupquote{float}}) \textendash{} target temperature of current step

\end{description}\end{quote}

\end{fulllineitems}

\index{get\_thermopower() (laboratory.Setup method)@\spxentry{get\_thermopower()}\spxextra{laboratory.Setup method}}

\begin{fulllineitems}
\phantomsection\label{\detokenize{laboratory:laboratory.Setup.get_thermopower}}\pysiglinewithargsret{\sphinxbfcode{\sphinxupquote{get\_thermopower}}}{}{}
Retrieves thermopower data from the DAQ and saves to Data structure and file
\begin{quote}\begin{description}
\item[{Returns}] \leavevmode
{[}thermistor, te1, te2, voltage{]}

\item[{Return type}] \leavevmode
list

\end{description}\end{quote}

\end{fulllineitems}

\index{load\_data() (laboratory.Setup method)@\spxentry{load\_data()}\spxextra{laboratory.Setup method}}

\begin{fulllineitems}
\phantomsection\label{\detokenize{laboratory:laboratory.Setup.load_data}}\pysiglinewithargsret{\sphinxbfcode{\sphinxupquote{load\_data}}}{\emph{filename}}{}
loads a previous data file for processing and analysis
\begin{quote}\begin{description}
\item[{Parameters}] \leavevmode
\sphinxstyleliteralstrong{\sphinxupquote{filename}} (\sphinxstyleliteralemphasis{\sphinxupquote{str}}) \textendash{} path to data file

\end{description}\end{quote}

\end{fulllineitems}

\index{load\_frequencies() (laboratory.Setup method)@\spxentry{load\_frequencies()}\spxextra{laboratory.Setup method}}

\begin{fulllineitems}
\phantomsection\label{\detokenize{laboratory:laboratory.Setup.load_frequencies}}\pysiglinewithargsret{\sphinxbfcode{\sphinxupquote{load\_frequencies}}}{\emph{min=20}, \emph{max=2000000}, \emph{n=50}, \emph{log=True}, \emph{filename=None}}{}
Loads an np.array of frequency values specified by either min, max and n or a file containing a list of frequencies specified by filename.
\begin{quote}\begin{description}
\item[{Parameters}] \leavevmode\begin{itemize}
\item {} 
\sphinxstyleliteralstrong{\sphinxupquote{filename}} (\sphinxstyleliteralemphasis{\sphinxupquote{str}}) \textendash{} name of file containing frequencies

\item {} 
\sphinxstyleliteralstrong{\sphinxupquote{n}} (\sphinxstyleliteralemphasis{\sphinxupquote{int}}) \textendash{} number of desired frequencies

\item {} 
\sphinxstyleliteralstrong{\sphinxupquote{min}} (\sphinxstyleliteralemphasis{\sphinxupquote{int}}\sphinxstyleliteralemphasis{\sphinxupquote{,}}\sphinxstyleliteralemphasis{\sphinxupquote{float}}) \textendash{} minimum frequency (Hz) - may not be below default value of 20 Hz

\item {} 
\sphinxstyleliteralstrong{\sphinxupquote{max}} (\sphinxstyleliteralemphasis{\sphinxupquote{int}}\sphinxstyleliteralemphasis{\sphinxupquote{,}}\sphinxstyleliteralemphasis{\sphinxupquote{float}}) \textendash{} maximum frequency (Hz) - may not exceed default value of 2*10\textasciicircum{}6 Hz

\item {} 
\sphinxstyleliteralstrong{\sphinxupquote{log}} (\sphinxstyleliteralemphasis{\sphinxupquote{boolean}}) \textendash{} specifies whether array is created in linear or log space. default to logspace

\end{itemize}

\item[{Example}] \leavevmode
\end{description}\end{quote}

\fvset{hllines={, ,}}%
\begin{sphinxVerbatim}[commandchars=\\\{\}]
\PYG{g+gp}{\PYGZgt{}\PYGZgt{}\PYGZgt{} }\PYG{n}{lab} \PYG{o}{=} \PYG{n}{Laboratory}\PYG{o}{.}\PYG{n}{Setup}\PYG{p}{(}\PYG{p}{)}
\PYG{g+gp}{\PYGZgt{}\PYGZgt{}\PYGZgt{} }\PYG{n}{lab}\PYG{o}{.}\PYG{n}{load\PYGZus{}frequencies}\PYG{p}{(}\PYG{n+nb}{min}\PYG{o}{=}\PYG{l+m+mi}{1000}\PYG{p}{,} \PYG{n+nb}{max}\PYG{o}{=}\PYG{l+m+mi}{10000}\PYG{p}{,} \PYG{n}{n}\PYG{o}{=}\PYG{l+m+mi}{10}\PYG{p}{)}
\PYG{g+gp}{\PYGZgt{}\PYGZgt{}\PYGZgt{} }\PYG{n+nb}{print}\PYG{p}{(}\PYG{n}{lab}\PYG{o}{.}\PYG{n}{data}\PYG{o}{.}\PYG{n}{freq}\PYG{p}{)}
\PYG{g+go}{[1000 2000 3000 4000 5000 6000 7000 8000 9000 10000]}
\PYG{g+gp}{\PYGZgt{}\PYGZgt{}\PYGZgt{} }\PYG{n}{lab}\PYG{o}{.}\PYG{n}{load\PYGZus{}frequencies}\PYG{p}{(}\PYG{n+nb}{min}\PYG{o}{=}\PYG{l+m+mi}{1000}\PYG{p}{,} \PYG{n+nb}{max}\PYG{o}{=}\PYG{l+m+mi}{10000}\PYG{p}{,} \PYG{n}{n}\PYG{o}{=}\PYG{l+m+mi}{10}\PYG{p}{,} \PYG{n}{log}\PYG{o}{=}\PYG{k+kc}{True}\PYG{p}{)}
\PYG{g+gp}{\PYGZgt{}\PYGZgt{}\PYGZgt{} }\PYG{n+nb}{print}\PYG{p}{(}\PYG{n}{lab}\PYG{o}{.}\PYG{n}{data}\PYG{o}{.}\PYG{n}{freq}\PYG{p}{)}
\PYG{g+go}{[1000 1291.55 1668.1 2154.43 2782.56 3593.81 4641.59 5994.84 7742.64 10000]}
\end{sphinxVerbatim}

\end{fulllineitems}

\index{load\_instruments() (laboratory.Setup method)@\spxentry{load\_instruments()}\spxextra{laboratory.Setup method}}

\begin{fulllineitems}
\phantomsection\label{\detokenize{laboratory:laboratory.Setup.load_instruments}}\pysiglinewithargsret{\sphinxbfcode{\sphinxupquote{load\_instruments}}}{}{}
Loads all the laboratory instruments. Called automatically when calling Setup() without a filename specified.
\begin{quote}\begin{description}
\item[{Returns}] \leavevmode
lcr, daq, mfc, furnace, motor

\item[{Return type}] \leavevmode
instrument objects

\end{description}\end{quote}

\end{fulllineitems}

\index{preflight\_checklist() (laboratory.Setup method)@\spxentry{preflight\_checklist()}\spxextra{laboratory.Setup method}}

\begin{fulllineitems}
\phantomsection\label{\detokenize{laboratory:laboratory.Setup.preflight_checklist}}\pysiglinewithargsret{\sphinxbfcode{\sphinxupquote{preflight\_checklist}}}{\emph{controlfile}}{}
Conducts necessary checks before running an experiment abs
\begin{quote}\begin{description}
\item[{Parameters}] \leavevmode
\sphinxstyleliteralstrong{\sphinxupquote{controlfile}} (\sphinxstyleliteralemphasis{\sphinxupquote{string}}) \textendash{} name of control file for the experiment

\end{description}\end{quote}

\end{fulllineitems}

\index{reconnect() (laboratory.Setup method)@\spxentry{reconnect()}\spxextra{laboratory.Setup method}}

\begin{fulllineitems}
\phantomsection\label{\detokenize{laboratory:laboratory.Setup.reconnect}}\pysiglinewithargsret{\sphinxbfcode{\sphinxupquote{reconnect}}}{}{}
Attempts to reconnect to any instruments that have been disconnected

\end{fulllineitems}

\index{restart\_from\_backup() (laboratory.Setup method)@\spxentry{restart\_from\_backup()}\spxextra{laboratory.Setup method}}

\begin{fulllineitems}
\phantomsection\label{\detokenize{laboratory:laboratory.Setup.restart_from_backup}}\pysiglinewithargsret{\sphinxbfcode{\sphinxupquote{restart\_from\_backup}}}{}{}
TODO - reload an aborted experiment and pick up where it left off

\end{fulllineitems}

\index{run() (laboratory.Setup method)@\spxentry{run()}\spxextra{laboratory.Setup method}}

\begin{fulllineitems}
\phantomsection\label{\detokenize{laboratory:laboratory.Setup.run}}\pysiglinewithargsret{\sphinxbfcode{\sphinxupquote{run}}}{\emph{controlfile=False}}{}
starts a new set of measurements. requires a control file that contains
specific instruction for the instruments to follow. see the tutorial section
for help setting up a control file.
\begin{quote}\begin{description}
\item[{Parameters}] \leavevmode
\sphinxstyleliteralstrong{\sphinxupquote{controlfile}} (\sphinxstyleliteralemphasis{\sphinxupquote{str}}) \textendash{} path to control file

\end{description}\end{quote}

\end{fulllineitems}

\index{save\_data() (laboratory.Setup method)@\spxentry{save\_data()}\spxextra{laboratory.Setup method}}

\begin{fulllineitems}
\phantomsection\label{\detokenize{laboratory:laboratory.Setup.save_data}}\pysiglinewithargsret{\sphinxbfcode{\sphinxupquote{save\_data}}}{\emph{val\_type}, \emph{vals}, \emph{gastype=None}}{}
Takes input values and saves to both the current Data object and an external file
\begin{quote}\begin{description}
\item[{Parameters}] \leavevmode\begin{itemize}
\item {} 
\sphinxstyleliteralstrong{\sphinxupquote{val\_type}} (\sphinxstyleliteralemphasis{\sphinxupquote{str}}) \textendash{} type of measurement being saved

\item {} 
\sphinxstyleliteralstrong{\sphinxupquote{vals}} (\sphinxstyleliteralemphasis{\sphinxupquote{list}}) \textendash{} the required values suitable to that specified by val\_type

\item {} 
\sphinxstyleliteralstrong{\sphinxupquote{gastype}} (\sphinxstyleliteralemphasis{\sphinxupquote{str}}) \textendash{} {[}optional{]} required when saving gas data

\end{itemize}

\end{description}\end{quote}

\end{fulllineitems}

\index{set\_fugacity() (laboratory.Setup method)@\spxentry{set\_fugacity()}\spxextra{laboratory.Setup method}}

\begin{fulllineitems}
\phantomsection\label{\detokenize{laboratory:laboratory.Setup.set_fugacity}}\pysiglinewithargsret{\sphinxbfcode{\sphinxupquote{set\_fugacity}}}{\emph{buffer}, \emph{offset}, \emph{gas\_type}}{}
Sets the correct gas ratio for the given buffer. Percentage offset from a given buffer can be specified by ‘offset’. Type of gas to be used for calculations is specified by gas\_type.
\begin{quote}\begin{description}
\item[{Parameters}] \leavevmode\begin{itemize}
\item {} 
\sphinxstyleliteralstrong{\sphinxupquote{buffer}} (\sphinxstyleliteralemphasis{\sphinxupquote{str}}) \textendash{} buffer type (see table for input options)

\item {} 
\sphinxstyleliteralstrong{\sphinxupquote{offset}} (\sphinxstyleliteralemphasis{\sphinxupquote{float}}\sphinxstyleliteralemphasis{\sphinxupquote{, }}\sphinxstyleliteralemphasis{\sphinxupquote{int}}) \textendash{} percentage offset from specified buffer

\item {} 
\sphinxstyleliteralstrong{\sphinxupquote{gas\_type}} \textendash{} gas type to use for calculating ratio - can be either ‘h2’ or ‘co’

\end{itemize}

\end{description}\end{quote}

\end{fulllineitems}

\index{shut\_down() (laboratory.Setup method)@\spxentry{shut\_down()}\spxextra{laboratory.Setup method}}

\begin{fulllineitems}
\phantomsection\label{\detokenize{laboratory:laboratory.Setup.shut_down}}\pysiglinewithargsret{\sphinxbfcode{\sphinxupquote{shut\_down}}}{}{}
Returns the furnace to a safe temperature and closes ports to both the DAQ and LCR. (TODO need to close ports to motor and furnace)

\end{fulllineitems}


\end{fulllineitems}



\chapter{Module Drivers}
\label{\detokenize{drivers:module-drivers}}\label{\detokenize{drivers:module-drivers}}\label{\detokenize{drivers::doc}}\index{drivers (module)@\spxentry{drivers}\spxextra{module}}
Contains drivers for each instrument in use in the laboratory.
\index{AlicatController (class in drivers)@\spxentry{AlicatController}\spxextra{class in drivers}}

\begin{fulllineitems}
\phantomsection\label{\detokenize{drivers:drivers.AlicatController}}\pysiglinewithargsret{\sphinxbfcode{\sphinxupquote{class }}\sphinxcode{\sphinxupquote{drivers.}}\sphinxbfcode{\sphinxupquote{AlicatController}}}{\emph{port}, \emph{address}}{}
Driver for an individual Mass Flow Controller.

\begin{sphinxadmonition}{note}{Note:}
not called directly - access is from within :class:’\textasciitilde{}Drivers.MFC’
\end{sphinxadmonition}


\begin{savenotes}\sphinxattablestart
\centering
\begin{tabulary}{\linewidth}[t]{|T|T|}
\hline
\sphinxstyletheadfamily 
Methods
&\sphinxstyletheadfamily 
message
\\
\hline
get\_massflow
&
gets massflow from controller
\\
\hline
set\_massflow
&
sets massflow on controller
\\
\hline
get\_pressure
&
gets pressure from controller
\\
\hline
set\_pressure
&
sets massflow on controller
\\
\hline
get\_temp
&
gets pressure from controller
\\
\hline
get\_vol\_flow
&
gets volumetric flow from controller
\\
\hline
get\_setpoint
&
gets current set point from controller
\\
\hline
reset
&
resets the device
\\
\hline
\end{tabulary}
\par
\sphinxattableend\end{savenotes}

*see FlowController for more methods
\index{massflow() (drivers.AlicatController method)@\spxentry{massflow()}\spxextra{drivers.AlicatController method}}

\begin{fulllineitems}
\phantomsection\label{\detokenize{drivers:drivers.AlicatController.massflow}}\pysiglinewithargsret{\sphinxbfcode{\sphinxupquote{massflow}}}{\emph{value=None}}{}
Get or set the massflow of the appropriate flowmeter
\begin{quote}\begin{description}
\item[{Parameters}] \leavevmode
\sphinxstyleliteralstrong{\sphinxupquote{value}} (\sphinxstyleliteralemphasis{\sphinxupquote{float}}) \textendash{} desired massflow value

\end{description}\end{quote}

\end{fulllineitems}

\index{pressure() (drivers.AlicatController method)@\spxentry{pressure()}\spxextra{drivers.AlicatController method}}

\begin{fulllineitems}
\phantomsection\label{\detokenize{drivers:drivers.AlicatController.pressure}}\pysiglinewithargsret{\sphinxbfcode{\sphinxupquote{pressure}}}{\emph{value=None}}{}
Get or set pressure of the appropriate flowmeter
\begin{quote}\begin{description}
\item[{Parameters}] \leavevmode
\sphinxstyleliteralstrong{\sphinxupquote{value}} (\sphinxstyleliteralemphasis{\sphinxupquote{float}}) \textendash{} desired massflow value

\end{description}\end{quote}

\end{fulllineitems}

\index{reset() (drivers.AlicatController method)@\spxentry{reset()}\spxextra{drivers.AlicatController method}}

\begin{fulllineitems}
\phantomsection\label{\detokenize{drivers:drivers.AlicatController.reset}}\pysiglinewithargsret{\sphinxbfcode{\sphinxupquote{reset}}}{}{}
sets the massflow to 0

\end{fulllineitems}

\index{setpoint() (drivers.AlicatController method)@\spxentry{setpoint()}\spxextra{drivers.AlicatController method}}

\begin{fulllineitems}
\phantomsection\label{\detokenize{drivers:drivers.AlicatController.setpoint}}\pysiglinewithargsret{\sphinxbfcode{\sphinxupquote{setpoint}}}{}{}
Gets the current set point of the appropriate flowmeter

\end{fulllineitems}

\index{temperature() (drivers.AlicatController method)@\spxentry{temperature()}\spxextra{drivers.AlicatController method}}

\begin{fulllineitems}
\phantomsection\label{\detokenize{drivers:drivers.AlicatController.temperature}}\pysiglinewithargsret{\sphinxbfcode{\sphinxupquote{temperature}}}{}{}
Gets the temperature of the appropriate flowmeter
:returns: gas temperature
:rtype: float

\end{fulllineitems}

\index{volume\_flow() (drivers.AlicatController method)@\spxentry{volume\_flow()}\spxextra{drivers.AlicatController method}}

\begin{fulllineitems}
\phantomsection\label{\detokenize{drivers:drivers.AlicatController.volume_flow}}\pysiglinewithargsret{\sphinxbfcode{\sphinxupquote{volume\_flow}}}{}{}
Gets the volumetric flow of the appropriate flowmeter

\end{fulllineitems}


\end{fulllineitems}

\index{DAQ (class in drivers)@\spxentry{DAQ}\spxextra{class in drivers}}

\begin{fulllineitems}
\phantomsection\label{\detokenize{drivers:drivers.DAQ}}\pysigline{\sphinxbfcode{\sphinxupquote{class }}\sphinxcode{\sphinxupquote{drivers.}}\sphinxbfcode{\sphinxupquote{DAQ}}}
Driver for the 34970A Data Acquisition / Data Logger Switch Unit


\begin{savenotes}\sphinxattablestart
\centering
\begin{tabulary}{\linewidth}[t]{|T|T|}
\hline
\sphinxstyletheadfamily 
Attributes
&\sphinxstyletheadfamily 
message
\\
\hline
maxtry
&
max number to attempt command
\\
\hline
status
&
whether the instrument is connected
\\
\hline
therm
&
specifies type of thermistor
\\
\hline
tref
&
‘101’ - channel for thermistor
\\
\hline
te1
&
‘104’ - channel for electrode 1
\\
\hline
te2
&
‘105’ - channel for electrode 2
\\
\hline
volt
&
‘103’ - channel for voltage measurements
\\
\hline
switch
&
‘205’,‘206’ - channels for switch between LCR and temp measurements
\\
\hline
address
&
computer port address
\\
\hline
\end{tabulary}
\par
\sphinxattableend\end{savenotes}


\begin{savenotes}\sphinxattablestart
\centering
\begin{tabulary}{\linewidth}[t]{|T|T|}
\hline
\sphinxstyletheadfamily 
Methods
&\sphinxstyletheadfamily 
message
\\
\hline
connect
&
attempt to connect to the LCR meter
\\
\hline
configure
&
configures device for measurements
\\
\hline
get\_temp
&
gets temperature from te1,te2 and tref
\\
\hline
get voltage
&
gets voltage measurement
\\
\hline
read\_errors
&
reads errors stored in the DAQ
\\
\hline
reset
&
resets the device
\\
\hline
shut\_down
&
shuts down the device
\\
\hline
toggle\_switch
&
switches configuration between temp and voltage
\\
\hline
\end{tabulary}
\par
\sphinxattableend\end{savenotes}

\begin{sphinxadmonition}{note}{Note:}
do not change class attributes unless the physical wiring has been changed within the DAQ
\end{sphinxadmonition}
\index{\_config\_temp() (drivers.DAQ method)@\spxentry{\_config\_temp()}\spxextra{drivers.DAQ method}}

\begin{fulllineitems}
\phantomsection\label{\detokenize{drivers:drivers.DAQ._config_temp}}\pysiglinewithargsret{\sphinxbfcode{\sphinxupquote{\_config\_temp}}}{}{}
Configures the thermistor (‘tref’) as 10,000 Ohm
Configures both electrodes (‘te1’ and ‘te2’) as S-type thermocouples
Sets units to degrees celsius

\end{fulllineitems}

\index{\_config\_volt() (drivers.DAQ method)@\spxentry{\_config\_volt()}\spxextra{drivers.DAQ method}}

\begin{fulllineitems}
\phantomsection\label{\detokenize{drivers:drivers.DAQ._config_volt}}\pysiglinewithargsret{\sphinxbfcode{\sphinxupquote{\_config\_volt}}}{}{}
Configures the voltage measurements

\end{fulllineitems}

\index{configure() (drivers.DAQ method)@\spxentry{configure()}\spxextra{drivers.DAQ method}}

\begin{fulllineitems}
\phantomsection\label{\detokenize{drivers:drivers.DAQ.configure}}\pysiglinewithargsret{\sphinxbfcode{\sphinxupquote{configure}}}{}{}
Configures the DAQ according to the current wiring

\end{fulllineitems}

\index{connect() (drivers.DAQ method)@\spxentry{connect()}\spxextra{drivers.DAQ method}}

\begin{fulllineitems}
\phantomsection\label{\detokenize{drivers:drivers.DAQ.connect}}\pysiglinewithargsret{\sphinxbfcode{\sphinxupquote{connect}}}{}{}
Connects to the DAQ

\end{fulllineitems}

\index{get\_temp() (drivers.DAQ method)@\spxentry{get\_temp()}\spxextra{drivers.DAQ method}}

\begin{fulllineitems}
\phantomsection\label{\detokenize{drivers:drivers.DAQ.get_temp}}\pysiglinewithargsret{\sphinxbfcode{\sphinxupquote{get\_temp}}}{}{}
Scans the thermistor and thermocouples for temperature readings
\begin{quote}\begin{description}
\item[{Returns}] \leavevmode
{[}tref,te1,te2{]}

\item[{Return type}] \leavevmode
list of floats (degrees Celsius)

\end{description}\end{quote}

\end{fulllineitems}

\index{get\_thermopower() (drivers.DAQ method)@\spxentry{get\_thermopower()}\spxextra{drivers.DAQ method}}

\begin{fulllineitems}
\phantomsection\label{\detokenize{drivers:drivers.DAQ.get_thermopower}}\pysiglinewithargsret{\sphinxbfcode{\sphinxupquote{get\_thermopower}}}{}{}
Collects both temperature and voltage data and returns a list

\end{fulllineitems}

\index{get\_voltage() (drivers.DAQ method)@\spxentry{get\_voltage()}\spxextra{drivers.DAQ method}}

\begin{fulllineitems}
\phantomsection\label{\detokenize{drivers:drivers.DAQ.get_voltage}}\pysiglinewithargsret{\sphinxbfcode{\sphinxupquote{get\_voltage}}}{}{}
Gets voltage across the sample from the DAQ
\begin{quote}\begin{description}
\item[{Returns}] \leavevmode
voltage

\item[{Return type}] \leavevmode
float

\end{description}\end{quote}

\end{fulllineitems}

\index{read\_errors() (drivers.DAQ method)@\spxentry{read\_errors()}\spxextra{drivers.DAQ method}}

\begin{fulllineitems}
\phantomsection\label{\detokenize{drivers:drivers.DAQ.read_errors}}\pysiglinewithargsret{\sphinxbfcode{\sphinxupquote{read\_errors}}}{}{}
Reads errors from the DAQ (unsure if working or not)

\end{fulllineitems}

\index{reset() (drivers.DAQ method)@\spxentry{reset()}\spxextra{drivers.DAQ method}}

\begin{fulllineitems}
\phantomsection\label{\detokenize{drivers:drivers.DAQ.reset}}\pysiglinewithargsret{\sphinxbfcode{\sphinxupquote{reset}}}{}{}
Resets the device

\end{fulllineitems}

\index{shutdown() (drivers.DAQ method)@\spxentry{shutdown()}\spxextra{drivers.DAQ method}}

\begin{fulllineitems}
\phantomsection\label{\detokenize{drivers:drivers.DAQ.shutdown}}\pysiglinewithargsret{\sphinxbfcode{\sphinxupquote{shutdown}}}{}{}
Shuts down the DAQ

\end{fulllineitems}

\index{toggle\_switch() (drivers.DAQ method)@\spxentry{toggle\_switch()}\spxextra{drivers.DAQ method}}

\begin{fulllineitems}
\phantomsection\label{\detokenize{drivers:drivers.DAQ.toggle_switch}}\pysiglinewithargsret{\sphinxbfcode{\sphinxupquote{toggle\_switch}}}{\emph{command}}{}
Opens or closes the switch to the lcr. Must be closed for impedance measurements and open for thermopower measurements.
\begin{quote}\begin{description}
\item[{Parameters}] \leavevmode
\sphinxstyleliteralstrong{\sphinxupquote{command}} (\sphinxstyleliteralemphasis{\sphinxupquote{str}}) \textendash{} either ‘thermo’ to make thermopower measurements or ‘impedance’ for impedance measurements

\end{description}\end{quote}

\end{fulllineitems}


\end{fulllineitems}

\index{Furnace (class in drivers)@\spxentry{Furnace}\spxextra{class in drivers}}

\begin{fulllineitems}
\phantomsection\label{\detokenize{drivers:drivers.Furnace}}\pysiglinewithargsret{\sphinxbfcode{\sphinxupquote{class }}\sphinxcode{\sphinxupquote{drivers.}}\sphinxbfcode{\sphinxupquote{Furnace}}}{\emph{ports=None}}{}
Driver for the Eurotherm 3216 Temperature Controller

\begin{sphinxadmonition}{note}{Note:}
units are in °C
\end{sphinxadmonition}


\begin{savenotes}\sphinxattablestart
\centering
\begin{tabulary}{\linewidth}[t]{|T|T|}
\hline
\sphinxstyletheadfamily 
Attributes
&\sphinxstyletheadfamily 
message
\\
\hline
maxtry
&
max number to attempt command
\\
\hline
default\_temp
&
revert to this temperature when resetting
\\
\hline
status
&
whether the instrument is connected
\\
\hline
address
&
computer port address
\\
\hline
\end{tabulary}
\par
\sphinxattableend\end{savenotes}


\begin{savenotes}\sphinxattablestart
\centering
\begin{tabulary}{\linewidth}[t]{|T|T|}
\hline
\sphinxstyletheadfamily 
Methods
&\sphinxstyletheadfamily 
message
\\
\hline
connect
&
attempt to connect to the LCR meter
\\
\hline
set\_temp
&
set temperature of furnace
\\
\hline
get\_temp
&
get temperature from furnace
\\
\hline
set\_heatrate
&
set heatrate of furnace
\\
\hline
get\_heatrate
&
get heatrate from furnace
\\
\hline
set\_other
&
set another parameter on furnace
\\
\hline
get\_other
&
get another parameter from furnace
\\
\hline
reset
&
resets the device
\\
\hline
\end{tabulary}
\par
\sphinxattableend\end{savenotes}
\index{\_connect() (drivers.Furnace method)@\spxentry{\_connect()}\spxextra{drivers.Furnace method}}

\begin{fulllineitems}
\phantomsection\label{\detokenize{drivers:drivers.Furnace._connect}}\pysiglinewithargsret{\sphinxbfcode{\sphinxupquote{\_connect}}}{\emph{ports}}{}
Attempts connection to the furnace through each port in ports. Stops searching when connection is successful
\begin{quote}\begin{description}
\item[{Parameters}] \leavevmode
\sphinxstyleliteralstrong{\sphinxupquote{ports}} (\sphinxstyleliteralemphasis{\sphinxupquote{list}}) \textendash{} names of available serial ports

\end{description}\end{quote}

\end{fulllineitems}

\index{heating\_rate() (drivers.Furnace method)@\spxentry{heating\_rate()}\spxextra{drivers.Furnace method}}

\begin{fulllineitems}
\phantomsection\label{\detokenize{drivers:drivers.Furnace.heating_rate}}\pysiglinewithargsret{\sphinxbfcode{\sphinxupquote{heating\_rate}}}{\emph{heat\_rate=None}, \emph{address=35}}{}
Sets the desired heating rate of furnace.
Modbus address - 35
\begin{quote}\begin{description}
\item[{Parameters}] \leavevmode
\sphinxstyleliteralstrong{\sphinxupquote{heat\_rate}} (\sphinxstyleliteralemphasis{\sphinxupquote{float}}\sphinxstyleliteralemphasis{\sphinxupquote{, }}\sphinxstyleliteralemphasis{\sphinxupquote{int}}) \textendash{} heating rate in °C/min

\item[{Returns}] \leavevmode
True if succesful, False if not

\item[{Return type}] \leavevmode
Boolean

\end{description}\end{quote}

\end{fulllineitems}

\index{indicated() (drivers.Furnace method)@\spxentry{indicated()}\spxextra{drivers.Furnace method}}

\begin{fulllineitems}
\phantomsection\label{\detokenize{drivers:drivers.Furnace.indicated}}\pysiglinewithargsret{\sphinxbfcode{\sphinxupquote{indicated}}}{\emph{address=1}}{}
Query current temperature of furnace.
Modbus address - 1
\begin{quote}\begin{description}
\item[{Returns}] \leavevmode
Temperature in °C if succesful, else False

\item[{Return type}] \leavevmode
float/boolean

\end{description}\end{quote}

\end{fulllineitems}

\index{other() (drivers.Furnace method)@\spxentry{other()}\spxextra{drivers.Furnace method}}

\begin{fulllineitems}
\phantomsection\label{\detokenize{drivers:drivers.Furnace.other}}\pysiglinewithargsret{\sphinxbfcode{\sphinxupquote{other}}}{\emph{address}, \emph{value=None}}{}
set value at specified modbus address.
\begin{quote}\begin{description}
\item[{Parameters}] \leavevmode\begin{itemize}
\item {} 
\sphinxstyleliteralstrong{\sphinxupquote{modbus\_address}} (\sphinxstyleliteralemphasis{\sphinxupquote{float}}\sphinxstyleliteralemphasis{\sphinxupquote{, }}\sphinxstyleliteralemphasis{\sphinxupquote{int}}) \textendash{} see furnace manual for adresses

\item {} 
\sphinxstyleliteralstrong{\sphinxupquote{val}} (\sphinxstyleliteralemphasis{\sphinxupquote{float}}\sphinxstyleliteralemphasis{\sphinxupquote{, }}\sphinxstyleliteralemphasis{\sphinxupquote{int}}) \textendash{} value to be sent to the furnace

\end{itemize}

\item[{Returns}] \leavevmode
True if succesful, False if not

\item[{Return type}] \leavevmode
Boolean

\end{description}\end{quote}

\end{fulllineitems}

\index{reset() (drivers.Furnace method)@\spxentry{reset()}\spxextra{drivers.Furnace method}}

\begin{fulllineitems}
\phantomsection\label{\detokenize{drivers:drivers.Furnace.reset}}\pysiglinewithargsret{\sphinxbfcode{\sphinxupquote{reset}}}{}{}
resets the furnace to default temperature

\end{fulllineitems}

\index{setpoint\_1() (drivers.Furnace method)@\spxentry{setpoint\_1()}\spxextra{drivers.Furnace method}}

\begin{fulllineitems}
\phantomsection\label{\detokenize{drivers:drivers.Furnace.setpoint_1}}\pysiglinewithargsret{\sphinxbfcode{\sphinxupquote{setpoint\_1}}}{\emph{temperature=None}, \emph{address=24}}{}
Sets target temperature of furnace.
Modbus address - 24
\begin{quote}\begin{description}
\item[{Parameters}] \leavevmode
\sphinxstyleliteralstrong{\sphinxupquote{temp}} (\sphinxstyleliteralemphasis{\sphinxupquote{float}}\sphinxstyleliteralemphasis{\sphinxupquote{, }}\sphinxstyleliteralemphasis{\sphinxupquote{int}}) \textendash{} temperature in °C

\item[{Returns}] \leavevmode
True if succesful, False if not

\item[{Return type}] \leavevmode
Boolean

\end{description}\end{quote}

\end{fulllineitems}

\index{setpoint\_2() (drivers.Furnace method)@\spxentry{setpoint\_2()}\spxextra{drivers.Furnace method}}

\begin{fulllineitems}
\phantomsection\label{\detokenize{drivers:drivers.Furnace.setpoint_2}}\pysiglinewithargsret{\sphinxbfcode{\sphinxupquote{setpoint\_2}}}{\emph{temperature=None}, \emph{address=25}}{}
Sets target temperature of furnace.
Modbus address - 24
\begin{quote}\begin{description}
\item[{Parameters}] \leavevmode
\sphinxstyleliteralstrong{\sphinxupquote{temp}} (\sphinxstyleliteralemphasis{\sphinxupquote{float}}\sphinxstyleliteralemphasis{\sphinxupquote{, }}\sphinxstyleliteralemphasis{\sphinxupquote{int}}) \textendash{} temperature in °C

\item[{Returns}] \leavevmode
True if succesful, False if not

\item[{Return type}] \leavevmode
Boolean

\end{description}\end{quote}

\end{fulllineitems}


\end{fulllineitems}

\index{LCR (class in drivers)@\spxentry{LCR}\spxextra{class in drivers}}

\begin{fulllineitems}
\phantomsection\label{\detokenize{drivers:drivers.LCR}}\pysigline{\sphinxbfcode{\sphinxupquote{class }}\sphinxcode{\sphinxupquote{drivers.}}\sphinxbfcode{\sphinxupquote{LCR}}}
Driver for the E4980A Precision LCR Meter, 20 Hz to 2 MHz


\begin{savenotes}\sphinxattablestart
\centering
\begin{tabulary}{\linewidth}[t]{|T|T|}
\hline
\sphinxstyletheadfamily 
Attributes
&\sphinxstyletheadfamily 
message
\\
\hline
maxtry
&
max number to attempt command
\\
\hline
status
&
whether the instrument is connected
\\
\hline
address
&
port name
\\
\hline
\end{tabulary}
\par
\sphinxattableend\end{savenotes}


\begin{savenotes}\sphinxattablestart
\centering
\begin{tabulary}{\linewidth}[t]{|T|T|}
\hline
\sphinxstyletheadfamily 
Methods
&\sphinxstyletheadfamily 
message
\\
\hline
connect
&
attempt to connect to the LCR meter
\\
\hline
configure
&
configures device for measurements
\\
\hline
write\_freq
&
transfers desired frequencies to the LCR meter
\\
\hline
trigger
&
gets impedance for one specified frequency
\\
\hline
get\_complexZ
&
retrieves complex impedance from the device
\\
\hline
reset
&
resets the device
\\
\hline
\end{tabulary}
\par
\sphinxattableend\end{savenotes}
\index{\_connect() (drivers.LCR method)@\spxentry{\_connect()}\spxextra{drivers.LCR method}}

\begin{fulllineitems}
\phantomsection\label{\detokenize{drivers:drivers.LCR._connect}}\pysiglinewithargsret{\sphinxbfcode{\sphinxupquote{\_connect}}}{}{}
Connects to the LCR meter

\end{fulllineitems}

\index{\_set\_continuous() (drivers.LCR method)@\spxentry{\_set\_continuous()}\spxextra{drivers.LCR method}}

\begin{fulllineitems}
\phantomsection\label{\detokenize{drivers:drivers.LCR._set_continuous}}\pysiglinewithargsret{\sphinxbfcode{\sphinxupquote{\_set\_continuous}}}{\emph{mode='ON'}}{}
Allows the LCR meter to auto change state from idle to ‘wait for trigger’

\end{fulllineitems}

\index{\_set\_format() (drivers.LCR method)@\spxentry{\_set\_format()}\spxextra{drivers.LCR method}}

\begin{fulllineitems}
\phantomsection\label{\detokenize{drivers:drivers.LCR._set_format}}\pysiglinewithargsret{\sphinxbfcode{\sphinxupquote{\_set\_format}}}{\emph{mode='ascii'}}{}
Sets the format type of the LCR meter. Defaults to ascii. TODO - allow for other format types

\end{fulllineitems}

\index{\_set\_source() (drivers.LCR method)@\spxentry{\_set\_source()}\spxextra{drivers.LCR method}}

\begin{fulllineitems}
\phantomsection\label{\detokenize{drivers:drivers.LCR._set_source}}\pysiglinewithargsret{\sphinxbfcode{\sphinxupquote{\_set\_source}}}{\emph{mode='remote'}}{}
Sets up the LCR meter to expect a trigger from a remote source

\end{fulllineitems}

\index{configure() (drivers.LCR method)@\spxentry{configure()}\spxextra{drivers.LCR method}}

\begin{fulllineitems}
\phantomsection\label{\detokenize{drivers:drivers.LCR.configure}}\pysiglinewithargsret{\sphinxbfcode{\sphinxupquote{configure}}}{\emph{freq}}{}
Appropriately configures the LCR meter for measurements

\end{fulllineitems}

\index{display() (drivers.LCR method)@\spxentry{display()}\spxextra{drivers.LCR method}}

\begin{fulllineitems}
\phantomsection\label{\detokenize{drivers:drivers.LCR.display}}\pysiglinewithargsret{\sphinxbfcode{\sphinxupquote{display}}}{\emph{mode=None}}{}
Sets the LCR meter to display frequencies as a list

\end{fulllineitems}

\index{function() (drivers.LCR method)@\spxentry{function()}\spxextra{drivers.LCR method}}

\begin{fulllineitems}
\phantomsection\label{\detokenize{drivers:drivers.LCR.function}}\pysiglinewithargsret{\sphinxbfcode{\sphinxupquote{function}}}{\emph{mode='impedance'}}{}
Sets up the LCR meter for complex impedance measurements

\end{fulllineitems}

\index{get\_complexZ() (drivers.LCR method)@\spxentry{get\_complexZ()}\spxextra{drivers.LCR method}}

\begin{fulllineitems}
\phantomsection\label{\detokenize{drivers:drivers.LCR.get_complexZ}}\pysiglinewithargsret{\sphinxbfcode{\sphinxupquote{get\_complexZ}}}{}{}
Collects complex impedance from the LCR meter

\end{fulllineitems}

\index{list\_mode() (drivers.LCR method)@\spxentry{list\_mode()}\spxextra{drivers.LCR method}}

\begin{fulllineitems}
\phantomsection\label{\detokenize{drivers:drivers.LCR.list_mode}}\pysiglinewithargsret{\sphinxbfcode{\sphinxupquote{list\_mode}}}{\emph{mode=None}}{}
Instructs LCR meter to take a single measurement per trigger

\end{fulllineitems}

\index{reset() (drivers.LCR method)@\spxentry{reset()}\spxextra{drivers.LCR method}}

\begin{fulllineitems}
\phantomsection\label{\detokenize{drivers:drivers.LCR.reset}}\pysiglinewithargsret{\sphinxbfcode{\sphinxupquote{reset}}}{}{}
Resets the LCR meter

\end{fulllineitems}

\index{shutdown() (drivers.LCR method)@\spxentry{shutdown()}\spxextra{drivers.LCR method}}

\begin{fulllineitems}
\phantomsection\label{\detokenize{drivers:drivers.LCR.shutdown}}\pysiglinewithargsret{\sphinxbfcode{\sphinxupquote{shutdown}}}{}{}
Resets the LCR meter and closes the serial port

\end{fulllineitems}

\index{trigger() (drivers.LCR method)@\spxentry{trigger()}\spxextra{drivers.LCR method}}

\begin{fulllineitems}
\phantomsection\label{\detokenize{drivers:drivers.LCR.trigger}}\pysiglinewithargsret{\sphinxbfcode{\sphinxupquote{trigger}}}{}{}
Triggers the next measurement

\end{fulllineitems}

\index{write\_freq() (drivers.LCR method)@\spxentry{write\_freq()}\spxextra{drivers.LCR method}}

\begin{fulllineitems}
\phantomsection\label{\detokenize{drivers:drivers.LCR.write_freq}}\pysiglinewithargsret{\sphinxbfcode{\sphinxupquote{write\_freq}}}{\emph{freq}}{}
Writes the desired frequencies to the LCR meter
\begin{quote}\begin{description}
\item[{Parameters}] \leavevmode
\sphinxstyleliteralstrong{\sphinxupquote{freq}} (\sphinxstyleliteralemphasis{\sphinxupquote{np.ndarray}}) \textendash{} array of frequencies

\end{description}\end{quote}

\end{fulllineitems}


\end{fulllineitems}

\index{MFC (class in drivers)@\spxentry{MFC}\spxextra{class in drivers}}

\begin{fulllineitems}
\phantomsection\label{\detokenize{drivers:drivers.MFC}}\pysigline{\sphinxbfcode{\sphinxupquote{class }}\sphinxcode{\sphinxupquote{drivers.}}\sphinxbfcode{\sphinxupquote{MFC}}}
Global driver for the Mass Flow Controllers

\begin{sphinxadmonition}{note}{Note:}
see AlicatController for methods to control individual gases
\end{sphinxadmonition}


\begin{savenotes}\sphinxattablestart
\centering
\begin{tabulary}{\linewidth}[t]{|T|T|}
\hline
\sphinxstyletheadfamily 
Attributes
&\sphinxstyletheadfamily 
message
\\
\hline
maxtry
&
max number to attempt command
\\
\hline
status
&
whether the instrument is connected
\\
\hline
co2
&
controls for the Carbon Dioxide (CO2) controller
\\
\hline
co\_a
&
controls for the coarse Carbon Monoxide (CO) controller
\\
\hline
co\_b
&
controls for the fine Carbon Monoxide (CO) controller
\\
\hline
h2
&
controls for the Hydrogen (H2) controller
\\
\hline
address
&
computer port address
\\
\hline
\end{tabulary}
\par
\sphinxattableend\end{savenotes}


\begin{savenotes}\sphinxattablestart
\centering
\begin{tabulary}{\linewidth}[t]{|T|T|}
\hline
\sphinxstyletheadfamily 
Methods
&\sphinxstyletheadfamily 
message
\\
\hline
close\_all
&
closes all controllers
\\
\hline
connect
&
attempt to connect to the LCR meter
\\
\hline
flush\_all
&
flushes data from the input/output buffer of all devices
\\
\hline
fugacity\_co
&
returns a ratio of CO2/CO to achieve desired oxygen fugacity
\\
\hline
fugacity\_h2
&
returns a ratio of H2/CO2 to achieve desired oxugen fugacity
\\
\hline
reset
&
resets the device
\\
\hline
\end{tabulary}
\par
\sphinxattableend\end{savenotes}
\begin{quote}\begin{description}
\item[{Example}] \leavevmode
\end{description}\end{quote}

\fvset{hllines={, ,}}%
\begin{sphinxVerbatim}[commandchars=\\\{\}]
\PYG{g+gp}{\PYGZgt{}\PYGZgt{}\PYGZgt{} }\PYG{k+kn}{import} \PYG{n+nn}{Drivers}
\PYG{g+gp}{\PYGZgt{}\PYGZgt{}\PYGZgt{} }\PYG{n}{mfc} \PYG{o}{=} \PYG{n}{Drivers}\PYG{o}{.}\PYG{n}{MFC}\PYG{p}{(}\PYG{p}{)}
\PYG{g+gp}{\PYGZgt{}\PYGZgt{}\PYGZgt{} }\PYG{n}{mfc}\PYG{o}{.}\PYG{n}{co2}\PYG{o}{.}\PYG{n}{get\PYGZus{}massflow}\PYG{p}{(}\PYG{p}{)}
\end{sphinxVerbatim}
\index{\_connect() (drivers.MFC method)@\spxentry{\_connect()}\spxextra{drivers.MFC method}}

\begin{fulllineitems}
\phantomsection\label{\detokenize{drivers:drivers.MFC._connect}}\pysiglinewithargsret{\sphinxbfcode{\sphinxupquote{\_connect}}}{}{}
Connects to the mass flow controllers

\end{fulllineitems}

\index{close\_all() (drivers.MFC method)@\spxentry{close\_all()}\spxextra{drivers.MFC method}}

\begin{fulllineitems}
\phantomsection\label{\detokenize{drivers:drivers.MFC.close_all}}\pysiglinewithargsret{\sphinxbfcode{\sphinxupquote{close\_all}}}{}{}
Closes all flow controllers

\end{fulllineitems}

\index{flush\_all() (drivers.MFC method)@\spxentry{flush\_all()}\spxextra{drivers.MFC method}}

\begin{fulllineitems}
\phantomsection\label{\detokenize{drivers:drivers.MFC.flush_all}}\pysiglinewithargsret{\sphinxbfcode{\sphinxupquote{flush\_all}}}{}{}
Flushes the input? buffer of all flow controllers

\end{fulllineitems}

\index{fo2\_buffer() (drivers.MFC method)@\spxentry{fo2\_buffer()}\spxextra{drivers.MFC method}}

\begin{fulllineitems}
\phantomsection\label{\detokenize{drivers:drivers.MFC.fo2_buffer}}\pysiglinewithargsret{\sphinxbfcode{\sphinxupquote{fo2\_buffer}}}{\emph{temp}, \emph{buffer}, \emph{pressure=1.01325}}{}
Calculates oxygen fugacity at a given temperature and fo2 buffer


\begin{savenotes}\sphinxattablestart
\centering
\begin{tabulary}{\linewidth}[t]{|T|T|}
\hline
\sphinxstyletheadfamily 
input options
&\sphinxstyletheadfamily 
type of fo2 buffer
\\
\hline
‘QFM’
&
quartz-fayalite-magnetite
\\
\hline
‘IW’
&
iron-wustite
\\
\hline
‘WM’
&
wustite-magnetite
\\
\hline
‘MH’
&
magnetite-hematite
\\
\hline
‘QIF’
&
quartz-iron-fayalite
\\
\hline
‘NNO’
&
nickel-nickel oxide
\\
\hline
‘MMO’
&
molyb
\\
\hline
‘CCO’
&
cobalt-cobalt oxide
\\
\hline
\end{tabulary}
\par
\sphinxattableend\end{savenotes}
\begin{quote}\begin{description}
\item[{Parameters}] \leavevmode\begin{itemize}
\item {} 
\sphinxstyleliteralstrong{\sphinxupquote{temp}} (\sphinxstyleliteralemphasis{\sphinxupquote{float}}\sphinxstyleliteralemphasis{\sphinxupquote{, }}\sphinxstyleliteralemphasis{\sphinxupquote{int}}) \textendash{} Temperature in u’°C’

\item {} 
\sphinxstyleliteralstrong{\sphinxupquote{buffer}} (\sphinxstyleliteralemphasis{\sphinxupquote{str}}) \textendash{} buffer type (see table for input options)

\item {} 
\sphinxstyleliteralstrong{\sphinxupquote{pressure}} (\sphinxstyleliteralemphasis{\sphinxupquote{float}}\sphinxstyleliteralemphasis{\sphinxupquote{, }}\sphinxstyleliteralemphasis{\sphinxupquote{int}}) \textendash{} pressure in bar (default: surface pressure)

\end{itemize}

\item[{Returns}] \leavevmode
log10 oxygen fugacity

\item[{Return type}] \leavevmode
float

\end{description}\end{quote}

\end{fulllineitems}

\index{fugacity\_co() (drivers.MFC method)@\spxentry{fugacity\_co()}\spxextra{drivers.MFC method}}

\begin{fulllineitems}
\phantomsection\label{\detokenize{drivers:drivers.MFC.fugacity_co}}\pysiglinewithargsret{\sphinxbfcode{\sphinxupquote{fugacity\_co}}}{\emph{fo2p}, \emph{temp}}{}
Calculates the ratio CO2/CO needed to maintain a constant oxygen fugacity at a given temperature.
\begin{quote}\begin{description}
\item[{Parameters}] \leavevmode\begin{itemize}
\item {} 
\sphinxstyleliteralstrong{\sphinxupquote{fo2p}} (\sphinxstyleliteralemphasis{\sphinxupquote{float}}\sphinxstyleliteralemphasis{\sphinxupquote{, }}\sphinxstyleliteralemphasis{\sphinxupquote{int}}) \textendash{} desired oxygen fugacity (log Pa)

\item {} 
\sphinxstyleliteralstrong{\sphinxupquote{temp}} (\sphinxstyleliteralemphasis{\sphinxupquote{float}}\sphinxstyleliteralemphasis{\sphinxupquote{, }}\sphinxstyleliteralemphasis{\sphinxupquote{int}}) \textendash{} temperature (u’°C)

\end{itemize}

\item[{Returns}] \leavevmode
CO2/CO ratio

\item[{Return type}] \leavevmode
float

\end{description}\end{quote}

\end{fulllineitems}

\index{fugacity\_h2() (drivers.MFC method)@\spxentry{fugacity\_h2()}\spxextra{drivers.MFC method}}

\begin{fulllineitems}
\phantomsection\label{\detokenize{drivers:drivers.MFC.fugacity_h2}}\pysiglinewithargsret{\sphinxbfcode{\sphinxupquote{fugacity\_h2}}}{\emph{fo2p}, \emph{temp}}{}
Calculates the ratio CO2/H2 needed to maintain a constant oxygen fugacity at a given temperature.
\begin{quote}\begin{description}
\item[{Parameters}] \leavevmode\begin{itemize}
\item {} 
\sphinxstyleliteralstrong{\sphinxupquote{fo2p}} (\sphinxstyleliteralemphasis{\sphinxupquote{float}}\sphinxstyleliteralemphasis{\sphinxupquote{, }}\sphinxstyleliteralemphasis{\sphinxupquote{int}}) \textendash{} desired oxygen fugacity (log Pa)

\item {} 
\sphinxstyleliteralstrong{\sphinxupquote{temp}} (\sphinxstyleliteralemphasis{\sphinxupquote{float}}\sphinxstyleliteralemphasis{\sphinxupquote{, }}\sphinxstyleliteralemphasis{\sphinxupquote{int}}) \textendash{} temperature (u’°C)

\end{itemize}

\item[{Returns}] \leavevmode
CO2/H2 ratio

\item[{Return type}] \leavevmode
float

\end{description}\end{quote}

\end{fulllineitems}

\index{reset() (drivers.MFC method)@\spxentry{reset()}\spxextra{drivers.MFC method}}

\begin{fulllineitems}
\phantomsection\label{\detokenize{drivers:drivers.MFC.reset}}\pysiglinewithargsret{\sphinxbfcode{\sphinxupquote{reset}}}{}{}
Resets all connected flow controllers to 0 massflow

\end{fulllineitems}


\end{fulllineitems}

\index{Motor (class in drivers)@\spxentry{Motor}\spxextra{class in drivers}}

\begin{fulllineitems}
\phantomsection\label{\detokenize{drivers:drivers.Motor}}\pysiglinewithargsret{\sphinxbfcode{\sphinxupquote{class }}\sphinxcode{\sphinxupquote{drivers.}}\sphinxbfcode{\sphinxupquote{Motor}}}{\emph{ports=None}}{}
Driver for the motor controlling the linear stage


\begin{savenotes}\sphinxattablestart
\centering
\begin{tabulary}{\linewidth}[t]{|T|T|}
\hline
\sphinxstyletheadfamily 
Attributes
&\sphinxstyletheadfamily 
message
\\
\hline
maxtry
&
max number to attempt command
\\
\hline
status
&
whether the instrument is connected
\\
\hline
home
&
approximate xpos where te1 == te2
\\
\hline
max\_xpos
&
maximum x-position of the stage
\\
\hline
address
&
computer port address
\\
\hline
\end{tabulary}
\par
\sphinxattableend\end{savenotes}


\begin{savenotes}\sphinxattablestart
\centering
\begin{tabulary}{\linewidth}[t]{|T|T|}
\hline
\sphinxstyletheadfamily 
Methods
&\sphinxstyletheadfamily 
message
\\
\hline
home
&
move to the center of the stage
\\
\hline
connect
&
attempt to connect to the LCR meter
\\
\hline
move
&
moves the stage the desired amount in mm
\\
\hline
get\_xpos
&
get the absolute position of the stage
\\
\hline
set\_xpos
&
moves the stage the desired amount in steps
\\
\hline
get\_speed
&
get the current speed of the stage
\\
\hline
set\_speed
&
sets the movement speed of the stage
\\
\hline
reset
&
resets the device
\\
\hline
test
&
sends stage on a test run
\\
\hline
\end{tabulary}
\par
\sphinxattableend\end{savenotes}
\index{\_connect() (drivers.Motor method)@\spxentry{\_connect()}\spxextra{drivers.Motor method}}

\begin{fulllineitems}
\phantomsection\label{\detokenize{drivers:drivers.Motor._connect}}\pysiglinewithargsret{\sphinxbfcode{\sphinxupquote{\_connect}}}{\emph{ports}}{}
attempts connection to the motor
\begin{quote}\begin{description}
\item[{Parameters}] \leavevmode
\sphinxstyleliteralstrong{\sphinxupquote{ports}} (\sphinxstyleliteralemphasis{\sphinxupquote{list}}\sphinxstyleliteralemphasis{\sphinxupquote{, }}\sphinxstyleliteralemphasis{\sphinxupquote{string}}) \textendash{} list of available ports

\end{description}\end{quote}

\end{fulllineitems}

\index{\_convertdisplacement() (drivers.Motor method)@\spxentry{\_convertdisplacement()}\spxextra{drivers.Motor method}}

\begin{fulllineitems}
\phantomsection\label{\detokenize{drivers:drivers.Motor._convertdisplacement}}\pysiglinewithargsret{\sphinxbfcode{\sphinxupquote{\_convertdisplacement}}}{\emph{displacement}}{}
Converts a positive or negative displacement (in mm) into a command recognisable by the motor

\end{fulllineitems}

\index{\_convertspeed() (drivers.Motor method)@\spxentry{\_convertspeed()}\spxextra{drivers.Motor method}}

\begin{fulllineitems}
\phantomsection\label{\detokenize{drivers:drivers.Motor._convertspeed}}\pysiglinewithargsret{\sphinxbfcode{\sphinxupquote{\_convertspeed}}}{\emph{speed}, \emph{default=True}}{}
Converts a speed given in mm/s into a command recognisable by the motor

\end{fulllineitems}

\index{center() (drivers.Motor method)@\spxentry{center()}\spxextra{drivers.Motor method}}

\begin{fulllineitems}
\phantomsection\label{\detokenize{drivers:drivers.Motor.center}}\pysiglinewithargsret{\sphinxbfcode{\sphinxupquote{center}}}{}{}
Moves stage to the absolute center

\end{fulllineitems}

\index{get\_speed() (drivers.Motor method)@\spxentry{get\_speed()}\spxextra{drivers.Motor method}}

\begin{fulllineitems}
\phantomsection\label{\detokenize{drivers:drivers.Motor.get_speed}}\pysiglinewithargsret{\sphinxbfcode{\sphinxupquote{get\_speed}}}{}{}
Gets the current speed of the motor
\begin{quote}\begin{description}
\item[{Returns}] \leavevmode
speed of motor

\item[{Return type}] \leavevmode
float

\end{description}\end{quote}

\end{fulllineitems}

\index{get\_xpos() (drivers.Motor method)@\spxentry{get\_xpos()}\spxextra{drivers.Motor method}}

\begin{fulllineitems}
\phantomsection\label{\detokenize{drivers:drivers.Motor.get_xpos}}\pysiglinewithargsret{\sphinxbfcode{\sphinxupquote{get\_xpos}}}{}{}
Gets the current position of the stage
\begin{quote}\begin{description}
\item[{Returns}] \leavevmode
x-position of stage

\item[{Return type}] \leavevmode
str

\end{description}\end{quote}

\end{fulllineitems}

\index{home() (drivers.Motor method)@\spxentry{home()}\spxextra{drivers.Motor method}}

\begin{fulllineitems}
\phantomsection\label{\detokenize{drivers:drivers.Motor.home}}\pysiglinewithargsret{\sphinxbfcode{\sphinxupquote{home}}}{}{}
Moves furnace to the center of the stage (x = 5000)

\end{fulllineitems}

\index{move() (drivers.Motor method)@\spxentry{move()}\spxextra{drivers.Motor method}}

\begin{fulllineitems}
\phantomsection\label{\detokenize{drivers:drivers.Motor.move}}\pysiglinewithargsret{\sphinxbfcode{\sphinxupquote{move}}}{\emph{displacement}}{}
Moves the stage in the positive or negative direction
\begin{quote}\begin{description}
\item[{Parameters}] \leavevmode
\sphinxstyleliteralstrong{\sphinxupquote{displacement}} (\sphinxstyleliteralemphasis{\sphinxupquote{float}}\sphinxstyleliteralemphasis{\sphinxupquote{, }}\sphinxstyleliteralemphasis{\sphinxupquote{int}}) \textendash{} positive or negative displacement {[}in mm{]}

\end{description}\end{quote}

\end{fulllineitems}

\index{reset() (drivers.Motor method)@\spxentry{reset()}\spxextra{drivers.Motor method}}

\begin{fulllineitems}
\phantomsection\label{\detokenize{drivers:drivers.Motor.reset}}\pysiglinewithargsret{\sphinxbfcode{\sphinxupquote{reset}}}{}{}
Resets the stage position so that xPos=0

\end{fulllineitems}

\index{set\_speed() (drivers.Motor method)@\spxentry{set\_speed()}\spxextra{drivers.Motor method}}

\begin{fulllineitems}
\phantomsection\label{\detokenize{drivers:drivers.Motor.set_speed}}\pysiglinewithargsret{\sphinxbfcode{\sphinxupquote{set\_speed}}}{\emph{motor\_speed}}{}
Sets the speed of the motor
\begin{quote}\begin{description}
\item[{Parameters}] \leavevmode
\sphinxstyleliteralstrong{\sphinxupquote{motor\_speed}} (\sphinxstyleliteralemphasis{\sphinxupquote{float}}\sphinxstyleliteralemphasis{\sphinxupquote{, }}\sphinxstyleliteralemphasis{\sphinxupquote{int}}) \textendash{} speed of the motor in mm/s

\end{description}\end{quote}

\end{fulllineitems}

\index{set\_xpos() (drivers.Motor method)@\spxentry{set\_xpos()}\spxextra{drivers.Motor method}}

\begin{fulllineitems}
\phantomsection\label{\detokenize{drivers:drivers.Motor.set_xpos}}\pysiglinewithargsret{\sphinxbfcode{\sphinxupquote{set\_xpos}}}{\emph{xpos}}{}
Moves the linear stage to an absolute x position
\begin{quote}\begin{description}
\item[{Parameters}] \leavevmode
\sphinxstyleliteralstrong{\sphinxupquote{xpos}} (\sphinxstyleliteralemphasis{\sphinxupquote{float}}\sphinxstyleliteralemphasis{\sphinxupquote{, }}\sphinxstyleliteralemphasis{\sphinxupquote{int}}) \textendash{} desired absolute position of stage in controller pulses

\end{description}\end{quote}

\end{fulllineitems}

\index{test() (drivers.Motor method)@\spxentry{test()}\spxextra{drivers.Motor method}}

\begin{fulllineitems}
\phantomsection\label{\detokenize{drivers:drivers.Motor.test}}\pysiglinewithargsret{\sphinxbfcode{\sphinxupquote{test}}}{}{}
Sends the motorised stage on a test run to ensure everything is working

\end{fulllineitems}


\end{fulllineitems}

\index{get\_ports() (in module drivers)@\spxentry{get\_ports()}\spxextra{in module drivers}}

\begin{fulllineitems}
\phantomsection\label{\detokenize{drivers:drivers.get_ports}}\pysiglinewithargsret{\sphinxcode{\sphinxupquote{drivers.}}\sphinxbfcode{\sphinxupquote{get\_ports}}}{}{}
Returns a list of available serial ports for connecting to the furnace and motor
\begin{quote}\begin{description}
\item[{Returns}] \leavevmode
list of available ports

\item[{Return type}] \leavevmode
list, str

\end{description}\end{quote}

\end{fulllineitems}



\chapter{Module Data}
\label{\detokenize{data:module-data}}\label{\detokenize{data:module-data}}\label{\detokenize{data::doc}}\index{data (module)@\spxentry{data}\spxextra{module}}\index{Data (class in data)@\spxentry{Data}\spxextra{class in data}}

\begin{fulllineitems}
\phantomsection\label{\detokenize{data:data.Data}}\pysiglinewithargsret{\sphinxbfcode{\sphinxupquote{class }}\sphinxcode{\sphinxupquote{data.}}\sphinxbfcode{\sphinxupquote{Data}}}{\emph{freq=None}, \emph{filename=None}}{}
Storage for all data collected during experiments. Data file are loaded into this object for processing and plotting


\begin{savenotes}\sphinxattablestart
\centering
\begin{tabulary}{\linewidth}[t]{|T|T|}
\hline
\sphinxstyletheadfamily 
Attributes
&\sphinxstyletheadfamily 
Description
\\
\hline
freq
&
array of frequencies for use by the LCR meter
\\
\hline
filename
&
name of the file being used
\\
\hline
time
&
times for each measurement
\\
\hline
thermo
&
stroes thermopower data
\\
\hline
gas
&
stores gas and fugacity data
\\
\hline
temp
&
stores temperature data
\\
\hline
imp
&
stores impedance data
\\
\hline
xpos
&
stores stage x position at each measurement
\\
\hline
\end{tabulary}
\par
\sphinxattableend\end{savenotes}
\begin{quote}\begin{description}
\item[{Example}] \leavevmode
\end{description}\end{quote}

\fvset{hllines={, ,}}%
\begin{sphinxVerbatim}[commandchars=\\\{\}]
\PYG{g+gp}{\PYGZgt{}\PYGZgt{}\PYGZgt{} }\PYG{n}{lab} \PYG{o}{=} \PYG{n}{Laboratory}\PYG{o}{.}\PYG{n}{Setup}\PYG{p}{(}\PYG{l+s+s1}{\PYGZsq{}}\PYG{l+s+s1}{somefile.dat}\PYG{l+s+s1}{\PYGZsq{}}\PYG{p}{)}
\PYG{g+gp}{\PYGZgt{}\PYGZgt{}\PYGZgt{} }\PYG{n+nb}{print}\PYG{p}{(}\PYG{n}{lab}\PYG{o}{.}\PYG{n}{data}\PYG{o}{.}\PYG{n}{temp}\PYG{o}{.}\PYG{n}{indicated}\PYG{p}{)}
\PYG{g+go}{[100,105,110,115,120]}
\end{sphinxVerbatim}

\end{fulllineitems}

\index{Gas (class in data)@\spxentry{Gas}\spxextra{class in data}}

\begin{fulllineitems}
\phantomsection\label{\detokenize{data:data.Gas}}\pysigline{\sphinxbfcode{\sphinxupquote{class }}\sphinxcode{\sphinxupquote{data.}}\sphinxbfcode{\sphinxupquote{Gas}}}
Stores the seperate gas data under one roof


\begin{savenotes}\sphinxattablestart
\centering
\begin{tabulary}{\linewidth}[t]{|T|T|}
\hline
\sphinxstyletheadfamily 
Attributes
&\sphinxstyletheadfamily 
Description
\\
\hline
h2
&
hydrogen flow rate
\\
\hline
co2
&
carbon dioxide flow rate
\\
\hline
co\_a
&
carbon monoxide corase flow rate
\\
\hline
co\_b
&
carbon monoxide corase flow rate
\\
\hline
\end{tabulary}
\par
\sphinxattableend\end{savenotes}

\end{fulllineitems}

\index{Impedance (class in data)@\spxentry{Impedance}\spxextra{class in data}}

\begin{fulllineitems}
\phantomsection\label{\detokenize{data:data.Impedance}}\pysigline{\sphinxbfcode{\sphinxupquote{class }}\sphinxcode{\sphinxupquote{data.}}\sphinxbfcode{\sphinxupquote{Impedance}}}
Stores complex impedance data


\begin{savenotes}\sphinxattablestart
\centering
\begin{tabulary}{\linewidth}[t]{|T|T|}
\hline
\sphinxstyletheadfamily 
Attributes
&\sphinxstyletheadfamily 
Description
\\
\hline
Z
&
impedance
\\
\hline
theta
&
phase angle
\\
\hline
\end{tabulary}
\par
\sphinxattableend\end{savenotes}

\end{fulllineitems}

\index{MFC\_data (class in data)@\spxentry{MFC\_data}\spxextra{class in data}}

\begin{fulllineitems}
\phantomsection\label{\detokenize{data:data.MFC_data}}\pysigline{\sphinxbfcode{\sphinxupquote{class }}\sphinxcode{\sphinxupquote{data.}}\sphinxbfcode{\sphinxupquote{MFC\_data}}}
Stores gas data for an individual mass flow controller

\end{fulllineitems}

\index{Temp (class in data)@\spxentry{Temp}\spxextra{class in data}}

\begin{fulllineitems}
\phantomsection\label{\detokenize{data:data.Temp}}\pysigline{\sphinxbfcode{\sphinxupquote{class }}\sphinxcode{\sphinxupquote{data.}}\sphinxbfcode{\sphinxupquote{Temp}}}
Stores furnace temperature data


\begin{savenotes}\sphinxattablestart
\centering
\begin{tabulary}{\linewidth}[t]{|T|T|}
\hline
\sphinxstyletheadfamily 
Attributes
&\sphinxstyletheadfamily 
Description
\\
\hline
target
&
target temperature of current cycle
\\
\hline
indicated
&
temperature indicated by furnace
\\
\hline
\end{tabulary}
\par
\sphinxattableend\end{savenotes}

\end{fulllineitems}

\index{Thermo (class in data)@\spxentry{Thermo}\spxextra{class in data}}

\begin{fulllineitems}
\phantomsection\label{\detokenize{data:data.Thermo}}\pysigline{\sphinxbfcode{\sphinxupquote{class }}\sphinxcode{\sphinxupquote{data.}}\sphinxbfcode{\sphinxupquote{Thermo}}}
Stores thermopower data


\begin{savenotes}\sphinxattablestart
\centering
\begin{tabulary}{\linewidth}[t]{|T|T|}
\hline
\sphinxstyletheadfamily 
Attributes
&\sphinxstyletheadfamily 
Description
\\
\hline
tref
&
temperature of the internal thermistor
\\
\hline
te1
&
temperature of electrode 1
\\
\hline
te2
&
temperature of electrode 2
\\
\hline
volt
&
voltage across the sample
\\
\hline
\end{tabulary}
\par
\sphinxattableend\end{savenotes}

\end{fulllineitems}

\index{parse\_datafile() (in module data)@\spxentry{parse\_datafile()}\spxextra{in module data}}

\begin{fulllineitems}
\phantomsection\label{\detokenize{data:data.parse_datafile}}\pysiglinewithargsret{\sphinxcode{\sphinxupquote{data.}}\sphinxbfcode{\sphinxupquote{parse\_datafile}}}{\emph{filename}}{}
Parses a text file and stores the data in the Lab.Data object
\begin{quote}\begin{description}
\item[{Parameters}] \leavevmode
\sphinxstyleliteralstrong{\sphinxupquote{filename}} (\sphinxstyleliteralemphasis{\sphinxupquote{str}}) \textendash{} name of the file to be parsed

\end{description}\end{quote}

\end{fulllineitems}



\chapter{Module Plotting}
\label{\detokenize{plotting:module-plotting}}\label{\detokenize{plotting:module-plotting}}\label{\detokenize{plotting::doc}}\index{plotting (module)@\spxentry{plotting}\spxextra{module}}
Created on Sat Apr 14 16:09:52 2018
@author: Sam
\index{LabPlots (class in plotting)@\spxentry{LabPlots}\spxextra{class in plotting}}

\begin{fulllineitems}
\phantomsection\label{\detokenize{plotting:plotting.LabPlots}}\pysiglinewithargsret{\sphinxbfcode{\sphinxupquote{class }}\sphinxcode{\sphinxupquote{plotting.}}\sphinxbfcode{\sphinxupquote{LabPlots}}}{\emph{data}}{}
Contains an assortment of useful plots for visualising self.data
\index{arrhenius() (plotting.LabPlots method)@\spxentry{arrhenius()}\spxextra{plotting.LabPlots method}}

\begin{fulllineitems}
\phantomsection\label{\detokenize{plotting:plotting.LabPlots.arrhenius}}\pysiglinewithargsret{\sphinxbfcode{\sphinxupquote{arrhenius}}}{}{}
Plots inverse temperature versus conductivity

\end{fulllineitems}

\index{cole() (plotting.LabPlots method)@\spxentry{cole()}\spxextra{plotting.LabPlots method}}

\begin{fulllineitems}
\phantomsection\label{\detokenize{plotting:plotting.LabPlots.cole}}\pysiglinewithargsret{\sphinxbfcode{\sphinxupquote{cole}}}{\emph{temp\_list}, \emph{start=0}, \emph{end=None}, \emph{fit=False}}{}
Creates a Cole-Cole plot (imaginary versus real impedance) at a given temperature. Finds the available self.data to the temperature specified by ‘temp’. A linear least squares circle fit can be added by setting fit=True.
\begin{quote}\begin{description}
\item[{Parameters}] \leavevmode
\sphinxstyleliteralstrong{\sphinxupquote{temp}} (\sphinxstyleliteralemphasis{\sphinxupquote{float/int}}) \textendash{} temperature in degrees C

\end{description}\end{quote}

\end{fulllineitems}

\index{cond\_fug() (plotting.LabPlots method)@\spxentry{cond\_fug()}\spxextra{plotting.LabPlots method}}

\begin{fulllineitems}
\phantomsection\label{\detokenize{plotting:plotting.LabPlots.cond_fug}}\pysiglinewithargsret{\sphinxbfcode{\sphinxupquote{cond\_fug}}}{}{}
Plots inverse temperature versus conductivity

\end{fulllineitems}

\index{cond\_time() (plotting.LabPlots method)@\spxentry{cond\_time()}\spxextra{plotting.LabPlots method}}

\begin{fulllineitems}
\phantomsection\label{\detokenize{plotting:plotting.LabPlots.cond_time}}\pysiglinewithargsret{\sphinxbfcode{\sphinxupquote{cond\_time}}}{}{}
Plots conductivity versus time

\end{fulllineitems}

\index{gas() (plotting.LabPlots method)@\spxentry{gas()}\spxextra{plotting.LabPlots method}}

\begin{fulllineitems}
\phantomsection\label{\detokenize{plotting:plotting.LabPlots.gas}}\pysiglinewithargsret{\sphinxbfcode{\sphinxupquote{gas}}}{}{}
Plots mass\_flow self.data for all gases versus time elapsed

\end{fulllineitems}

\index{imp\_diameter() (plotting.LabPlots method)@\spxentry{imp\_diameter()}\spxextra{plotting.LabPlots method}}

\begin{fulllineitems}
\phantomsection\label{\detokenize{plotting:plotting.LabPlots.imp_diameter}}\pysiglinewithargsret{\sphinxbfcode{\sphinxupquote{imp\_diameter}}}{}{}
Plots the impedance diameter versus time\_elapsed

\end{fulllineitems}

\index{temperature() (plotting.LabPlots method)@\spxentry{temperature()}\spxextra{plotting.LabPlots method}}

\begin{fulllineitems}
\phantomsection\label{\detokenize{plotting:plotting.LabPlots.temperature}}\pysiglinewithargsret{\sphinxbfcode{\sphinxupquote{temperature}}}{}{}
Plots furnace indicated and target temperature, thermocouple temperature and thermistor self.data versus time elapsed

\end{fulllineitems}

\index{voltage() (plotting.LabPlots method)@\spxentry{voltage()}\spxextra{plotting.LabPlots method}}

\begin{fulllineitems}
\phantomsection\label{\detokenize{plotting:plotting.LabPlots.voltage}}\pysiglinewithargsret{\sphinxbfcode{\sphinxupquote{voltage}}}{}{}
Plots voltage versus time

\end{fulllineitems}


\end{fulllineitems}

\index{\_calculate\_conductivity() (in module plotting)@\spxentry{\_calculate\_conductivity()}\spxextra{in module plotting}}

\begin{fulllineitems}
\phantomsection\label{\detokenize{plotting:plotting._calculate_conductivity}}\pysiglinewithargsret{\sphinxcode{\sphinxupquote{plotting.}}\sphinxbfcode{\sphinxupquote{\_calculate\_conductivity}}}{\emph{Z}, \emph{theta}}{}
conductivity = length / area * resistance

\end{fulllineitems}



\chapter{Module Utils}
\label{\detokenize{utils:module-utils}}\label{\detokenize{utils:module-utils}}\label{\detokenize{utils::doc}}\index{utils (module)@\spxentry{utils}\spxextra{module}}\index{check\_controlfile() (in module utils)@\spxentry{check\_controlfile()}\spxextra{in module utils}}

\begin{fulllineitems}
\phantomsection\label{\detokenize{utils:utils.check_controlfile}}\pysiglinewithargsret{\sphinxcode{\sphinxupquote{utils.}}\sphinxbfcode{\sphinxupquote{check\_controlfile}}}{\emph{controlfile}}{}
Checks to make sure the specified controlfile is a valid file that can be used by this program
\begin{quote}\begin{description}
\item[{Parameters}] \leavevmode
\sphinxstyleliteralstrong{\sphinxupquote{controlfile}} (\sphinxstyleliteralemphasis{\sphinxupquote{pd.DataFrame}}) \textendash{} a loaded control file

\end{description}\end{quote}

\end{fulllineitems}

\index{data\_logger() (in module utils)@\spxentry{data\_logger()}\spxextra{in module utils}}

\begin{fulllineitems}
\phantomsection\label{\detokenize{utils:utils.data_logger}}\pysiglinewithargsret{\sphinxcode{\sphinxupquote{utils.}}\sphinxbfcode{\sphinxupquote{data\_logger}}}{}{}
Sets up the data file in much the same way as the log file.
Data cannot be output to the console. Data file can be found in
/datafiles/

\end{fulllineitems}

\index{find\_center() (in module utils)@\spxentry{find\_center()}\spxextra{in module utils}}

\begin{fulllineitems}
\phantomsection\label{\detokenize{utils:utils.find_center}}\pysiglinewithargsret{\sphinxcode{\sphinxupquote{utils.}}\sphinxbfcode{\sphinxupquote{find\_center}}}{\emph{self}}{}
TODO - Attempts to place the sample at the center of the heat source such that
te1 = te2. untested.

\end{fulllineitems}

\index{furnace\_profile() (in module utils)@\spxentry{furnace\_profile()}\spxextra{in module utils}}

\begin{fulllineitems}
\phantomsection\label{\detokenize{utils:utils.furnace_profile}}\pysiglinewithargsret{\sphinxcode{\sphinxupquote{utils.}}\sphinxbfcode{\sphinxupquote{furnace\_profile}}}{\emph{self}}{}
Records the temperature of both electrodes (te1 and te2) as the sample is moved
from one end of the stage to the other. Used to find the center of the stage or the xpos of a desired temperature gradient when taking thermopower measurements.

\end{fulllineitems}

\index{lab\_logger() (in module utils)@\spxentry{lab\_logger()}\spxextra{in module utils}}

\begin{fulllineitems}
\phantomsection\label{\detokenize{utils:utils.lab_logger}}\pysiglinewithargsret{\sphinxcode{\sphinxupquote{utils.}}\sphinxbfcode{\sphinxupquote{lab\_logger}}}{\emph{name}}{}
Sets up logging messages for the laboratory. Sends to both a file and the console by default. Levels for both the file and console can be set to anything defined by the
python logging package (DEBUG,INFO,WARNING,ERROR,CRITICAL). Specified log level AND GREATER will be included. Logfiles can be found in /logfiles/

\end{fulllineitems}

\index{load\_frequencies() (in module utils)@\spxentry{load\_frequencies()}\spxextra{in module utils}}

\begin{fulllineitems}
\phantomsection\label{\detokenize{utils:utils.load_frequencies}}\pysiglinewithargsret{\sphinxcode{\sphinxupquote{utils.}}\sphinxbfcode{\sphinxupquote{load\_frequencies}}}{\emph{min}, \emph{max}, \emph{n}, \emph{log}, \emph{filename}}{}
Creates an np.array of frequency values specified by either min, max and n or a file containing a list of frequencies specified by filename

\end{fulllineitems}

\index{load\_obj() (in module utils)@\spxentry{load\_obj()}\spxextra{in module utils}}

\begin{fulllineitems}
\phantomsection\label{\detokenize{utils:utils.load_obj}}\pysiglinewithargsret{\sphinxcode{\sphinxupquote{utils.}}\sphinxbfcode{\sphinxupquote{load\_obj}}}{\emph{filename}}{}
Loads a .pkl file
\begin{quote}\begin{description}
\item[{Parameters}] \leavevmode
\sphinxstyleliteralstrong{\sphinxupquote{filename}} (\sphinxstyleliteralemphasis{\sphinxupquote{str}}) \textendash{} full path to file. must be a .pkl

\end{description}\end{quote}

\end{fulllineitems}

\index{save\_obj() (in module utils)@\spxentry{save\_obj()}\spxextra{in module utils}}

\begin{fulllineitems}
\phantomsection\label{\detokenize{utils:utils.save_obj}}\pysiglinewithargsret{\sphinxcode{\sphinxupquote{utils.}}\sphinxbfcode{\sphinxupquote{save\_obj}}}{\emph{obj}, \emph{filename}}{}
Saves an object instance as a .pkl file for later retrieval. Can be loaded again using :meth:’Utils.load\_obj’
\begin{quote}\begin{description}
\item[{Parameters}] \leavevmode\begin{itemize}
\item {} 
\sphinxstyleliteralstrong{\sphinxupquote{obj}} (\sphinxstyleliteralemphasis{\sphinxupquote{class}}) \textendash{} the object instance to be saved

\item {} 
\sphinxstyleliteralstrong{\sphinxupquote{filename}} (\sphinxstyleliteralemphasis{\sphinxupquote{str}}) \textendash{} name of file

\end{itemize}

\end{description}\end{quote}

\end{fulllineitems}

\index{send\_email() (in module utils)@\spxentry{send\_email()}\spxextra{in module utils}}

\begin{fulllineitems}
\phantomsection\label{\detokenize{utils:utils.send_email}}\pysiglinewithargsret{\sphinxcode{\sphinxupquote{utils.}}\sphinxbfcode{\sphinxupquote{send\_email}}}{\emph{toaddr}, \emph{message}, \emph{cc=False}, \emph{logfile=False}, \emph{datafile=False}}{}
Sends an email to the specified email address. logfile or datafile can
be attached if desired. used mainly for email updates on progress during
long measurement cycles. mailer is \sphinxhref{mailto:geophysicslabnotifications@gmail.com}{geophysicslabnotifications@gmail.com}.
\begin{quote}\begin{description}
\item[{Parameters}] \leavevmode\begin{itemize}
\item {} 
\sphinxstyleliteralstrong{\sphinxupquote{toaddr}} (\sphinxstyleliteralemphasis{\sphinxupquote{str}}) \textendash{} full email address of intended recipient

\item {} 
\sphinxstyleliteralstrong{\sphinxupquote{message}} (\sphinxstyleliteralemphasis{\sphinxupquote{str}}) \textendash{} message to include in email

\item {} 
\sphinxstyleliteralstrong{\sphinxupquote{cc}} (\sphinxstyleliteralemphasis{\sphinxupquote{str}}\sphinxstyleliteralemphasis{\sphinxupquote{,}}\sphinxstyleliteralemphasis{\sphinxupquote{list}}) \textendash{} email can be carbon copied to additional adresses in cc

\item {} 
\sphinxstyleliteralstrong{\sphinxupquote{logfile}} (\sphinxstyleliteralemphasis{\sphinxupquote{boolean}}) \textendash{} whether to attach the current logfile

\item {} 
\sphinxstyleliteralstrong{\sphinxupquote{datafile}} (\sphinxstyleliteralemphasis{\sphinxupquote{boolean}}) \textendash{} whether to attach the current datafile

\end{itemize}

\end{description}\end{quote}

\end{fulllineitems}



\chapter{Indices and tables}
\label{\detokenize{index:indices-and-tables}}\begin{itemize}
\item {} 
\DUrole{xref,std,std-ref}{genindex}

\item {} 
\DUrole{xref,std,std-ref}{modindex}

\item {} 
\DUrole{xref,std,std-ref}{search}

\end{itemize}


\renewcommand{\indexname}{Python Module Index}
\begin{sphinxtheindex}
\let\bigletter\sphinxstyleindexlettergroup
\bigletter{d}
\item\relax\sphinxstyleindexentry{data}\sphinxstyleindexpageref{data:\detokenize{module-data}}
\item\relax\sphinxstyleindexentry{drivers}\sphinxstyleindexpageref{drivers:\detokenize{module-drivers}}
\indexspace
\bigletter{l}
\item\relax\sphinxstyleindexentry{laboratory}\sphinxstyleindexpageref{laboratory:\detokenize{module-laboratory}}
\indexspace
\bigletter{p}
\item\relax\sphinxstyleindexentry{plotting}\sphinxstyleindexpageref{plotting:\detokenize{module-plotting}}
\indexspace
\bigletter{u}
\item\relax\sphinxstyleindexentry{utils}\sphinxstyleindexpageref{utils:\detokenize{module-utils}}
\end{sphinxtheindex}

\renewcommand{\indexname}{Index}
\printindex
\end{document}